\documentclass[10pt]{article}
\usepackage[utf8]{inputenc}
\usepackage[T1]{fontenc}
\usepackage{graphicx}
\usepackage[export]{adjustbox}
\graphicspath{ {./images/} }
\usepackage{hyperref}
\hypersetup{colorlinks=true, linkcolor=blue, filecolor=magenta, urlcolor=cyan,}
\urlstyle{same}
\usepackage{amsmath}
\usepackage{amsfonts}
\usepackage{amssymb}
\usepackage[version=4]{mhchem}
\usepackage{stmaryrd}

\title{Technical University of Moldova }


\author{"CLI App"}
\date{}


\begin{document}
\maketitle
\begin{center}
\includegraphics[max width=\textwidth]{2023_05_24_83517335a92415ee8a42g-1}
\end{center}

Faculty of Computers, Informatics and Microelectronics

Department of Software Engineering and Automation

Author: Ovadenco Igor gr. FAF-203

Supervisor: Alexei Șerșun

Chișinău 2023

\section{Task:}
\begin{enumerate}
  \item You have to write a command line program, using \href{go2web}{go2web} executable as a starting point;

  \item The program should implement at least the following CLI:

\end{enumerate}

go2web -u $\langle$ URL $>$ \# make an HTTP request to the specified URL and print the response

go2 web -s  \# make an HTTP request to search the term using your favorite search engine and print top 10 results

go2web -h \# show this help

\begin{enumerate}
  \setcounter{enumi}{2}
  \item The responses from request should be human-readable (e.g. no HTML tags in the output)
\end{enumerate}

\section{Repository: https://github.com/ovadenco/ProgramareaWEB.git}
\section{Results:}
The program allows users to execute different operations based on command-line arguments. The available options include:

\begin{itemize}
  \item $\quad-\mathrm{u}<U R L>:$ This option makes an HTTP request to the specified URL and prints the response. It uses the HttpClient class from the System.Net.Http namespace to perform the request.

  \item $\quad-s<$ search-term>: This option makes an HTTP request to search for the specified term using a favorite search engine. It constructs the search URL and passes it to the MakeHttpRequest method. The search term is provided as a command-line argument and is URL-encoded using Uri.EscapeDataString.

  \item $\quad-h:$ This option displays the help information, explaining the available command-line options and their usage.

\end{itemize}

The program includes helper methods such as MakeHttpRequest, Search, ShowHelp,

ExtractBodyContent, ExtractText, ExtractTextFromHtml, and RemoveHtmITags. These methods are responsible for tasks such as making HTTP requests, extracting the body content from HTML, removing HTML tags, and decoding HTML entities.

\begin{center}
\includegraphics[max width=\textwidth]{2023_05_24_83517335a92415ee8a42g-2}
\end{center}

\begin{center}
\includegraphics[max width=\textwidth]{2023_05_24_83517335a92415ee8a42g-3}
\end{center}

\begin{center}
\includegraphics[max width=\textwidth]{2023_05_24_83517335a92415ee8a42g-4}
\end{center}

\begin{center}
\includegraphics[max width=\textwidth]{2023_05_24_83517335a92415ee8a42g-5}
\end{center}

\section{Conclusion:}
In conclusion, the project presented is a C\# program that demonstrates various functionalities related to making HTTP requests, parsing HTML content, and performing text extraction. The program utilizes several namespaces, including System, System.Net.Http, System.Threading.Tasks, System.Text.RegularExpressions, System.Text, \href{http://System.Web}{System.Web}, System.Collections.Generic, and System.Linq. Overall, the project showcases the use of various C\# features and, manipulate HTML content, and perform text extraction.

\section{Bibliography:}
\href{https://wiki.bash-hackers.org/howto/getopts}{https://wiki.bash-hackers.org/howto/getopts} tutorial

\href{https://learn.microsoft.com/en-us/dotnet/api/system.text.regularexpressions?view=net-7.0}{https://learn.microsoft.com/en-us/dotnet/api/system.text.regularexpressions?view=net-7.0}

\href{https://learn.microsoft.com/en-us/dotnet/api/system.net.http?view=net-7.0}{https://learn.microsoft.com/en-us/dotnet/api/system.net.http?view=net-7.0}


\end{document}